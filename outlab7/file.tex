\documentclass[11pt,a4paper]{article}
\RequirePackage{amsmath}
\begin{document}
\title {\textbf{Software Systems Lab 7:\\
	TeX Lab}}
\author{Saiteja Talluri}
\maketitle
\newpage
\tableofcontents
\newpage
\section{About Me}
Hello, My name is Saiteja Talluri.  I am currently pursuing B.Tech in Computer Science Department at IIT Bomaby.This is a  L\textsuperscript{A}T\textsubscript{E}X document for the course \textbf{Software Systems Lab} with course id cs251.  I would like for this document  to  be  typesetted  perfectly  which  forces  me  to  you  use  L\textsuperscript{A}T\textsubscript{E}X.  L\textsuperscript{A}T\textsubscript{E}X uses  various  packages.   I  will  elaborate  about  them  in  the  following
subsections:

\subsection{graphicx package{\normalfont}}
This  package  is  used  to  import  tables,  and  figure  in  the  document. Our document type is article, and we are currently using 11pt font size, with a4 type paper, which is specified in the beginning in
\textless documentclass \textgreater.

\subsection{amssymb package{\normalfont}}
This package is used to import mathematical symbols in the document.  We
encapsulate the mathematical equations and symbols under \$, and they are
changed to maths symbols.
\section{Some History}
I am ancient creature dwelling on this planet now referred to as \^{a}\u{A}\'{Z}Earth\^{a}\u{A}\.{Z}. I have been existing since the past 150393894.5 years.  Do you see the use of a package above in the number mention in the document.  I have used something  to  enunciate  the  numbers  in  a  fashion  such  as  a  mathematical formulae. Let us all try to replicate the text provided in this document. \emph{P.S.:Please note that I am following the Section Title Noun Capitalization in the document. This would be followed in the rest of the document,
henceforth}
\section{Replication}
 minimal input file looks like following\\ \center{\textbackslash documentclass\{class\} \\ 

\textbackslash begin\{document\} \\
your  text ... \\
\textbackslash end \{document\}}

where the class is a valid document class for LaTeX.
\section{Opening and compiling Tex Document}
First create a .tex file using text editor such as Vi or Gedit or Kile.
\section{Starting and Ending}
\subsection{Compiling the LaTex Document{\normalfont}}
\section{section}
\section{Cross Reference}
\subsection{\textbackslash label\{key\}{\normalfont}}
\end{document}

